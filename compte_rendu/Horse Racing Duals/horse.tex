\documentclass{article}

\usepackage[utf8]{inputenc}
\usepackage[T1]{fontenc}
\usepackage[english,francais]{babel}
\usepackage{textcomp}
\usepackage{amsmath,amssymb}
\usepackage{lmodern}
\usepackage[a4paper]{geometry}
\usepackage{graphicx}
\usepackage{xcolor}
\usepackage{verbatim}
\usepackage{microtype}
\usepackage{hyperref}
\hypersetup{pdfstartview=XYZ}

\title{Compte rendu Horse Racing Duals}
\author{lilian HIAULT}
\date{17 mai 2019}

\begin{document}

\maketitle

\tableofcontents

\section*{Introduction}

<<Horse Racing Duals>> est un problème posé sur Codingame (\url{https://www.codingame.com/training/easy/horse-racing-duals}. Il faut afficher le plus faible écart de puissance entre deux chevaux parmi ceux présentés.

\section{Coment stocker les données}

On crée un tableau dans lequel on stocke les performances des chevaux.


\begin{verbatim}
int main()
{
    int n;
    int *tab = (int*) malloc(n*sizeof(int));
    scanf("%d", &n);
    for (int i = 0; i < n; i++) {
        int Pi;
        scanf("%d", &Pi);
        tab[i] = Pi;
    }
\end{verbatim}

\section{Trouver le plus petit écart}

\subsection{Trier la tableau}

Afin de calculer les écarts plus rapidement je trie la tableau grâce à l'algorithme proposé sur le sujet.

\begin{verbatim}
qsort(tab, n, sizeof(int ), compare);
\end{verbatim}

La fonction est d'abord définie au début du programme.

\begin{verbatim}
 int compare(void const *a, void const *b)
 {
     int const *pa = a;
     int const *pb = b;
     return *pa-*pb;
}
\end{verbatim}

\subsection{Calculer les écarts}

Une fois que le tableau est trié il suffit de calculer les écarts de puissance entre un cheval et celui qui est juste à côté.

\begin{verbatim}
    int j;
    int diff = tab[1] - tab[0];
    for (j=1; j<n; j++)
    {
        if ((tab[j] - tab[j-1]) <= diff)
        {
            diff = tab[j] - tab[j-1];
        }
    }
    printf("%d\n", diff);
    return 0;
}
\end{verbatim}

\section{Conclusion}

Le programme fonctionne dans les cas simples mais il n'a atteint un score que de 53\% sur Codingame.

\end{document}
